For the purposes of this report, SDEs can be conceptually viewed as differential equations
with an additional random term that introduces stochastic fluctuations to their dyamics. 

To illustrate, consider the deterministic differential equation:

\begin{equation*}
    dx = \mu(x(t),t)dt.
\end{equation*}

Here, $x(t)$ is a deterministic variable with a drift term $\mu(x(t), t)$, mapping $\mathbb{R} \times[0, \infty) \to \mathbb{R}$. 
By adding stochasticity through Brownian motion

\begin{equation}\label{eq:sde}
    dX = \mu(X(t),t)dt + \sigma(X(t), t)dW
\end{equation}

where $X(t)$ is now a stochastic process. Here, $\mu(X(t), t)dt$ continues to be the deterministic drift term, while
$\sigma(X_t, t)$ is known as the \textit{diffusion term}, scaling the stochastic fluctuations introduced by the 
Wiener process $W_t$. The increments of $W_t$ are normally distributed as follows:

\begin{equation*}
    dW \sim \mathcal{N}(0, dt).
\end{equation*}

By selecting a sufficiently small time step $\Delta t$, we discretise equation \eqref{eq:sde} to 
numerically approximate the differential as:

\begin{equation}\label{eq:differential_approximation}
    \Delta X = \mu(X(t), t)\Delta t + \sigma(X(t), t) \Delta W
\end{equation}

Equation \eqref{eq:differential_approximation} can be used in an iterative approach to obtain a 
sample trajectory of the SDE in equation \eqref{eq:sde}:

\begin{equation}\label{eq:forward_euler}
    X(t + \Delta t) = X(t) + \Delta X.
\end{equation}

This method is called the Euler-Maruyama method, a stochastic
analogue of the forward Euler method \cite{erban2020stochastic}. 

As $X(t)$ is a stochastic process, 
at any fixed time $t$, $X(t)$ has a distribution described by its 
probability distribution function (PDF).
To obtain the PDF of the random variable $X(t)$ at a given time $t$, two main approaches exist. Firstly, using
the Euler-Maruyama method one can generate
$n$ independent trajectories of the SDE. Provided $n$ is sufficiently large, one can construct 
an empirical density from the realisations of $X(t)$ at time $t$. This approach is called the Monte Carlo method, 
and its accuracy depends on the number of samples $n$. By the central limit theorem, the error in 
estimating the PDF decreases like:

\begin{equation*} 
    \text{Error} \sim \mathcal{O}(n^{-1/2}). 
\end{equation*}


This means that to halve the error we must quadruple the number of samples. This can make Monte Carlo methods
computationally expensive for high precision estimates. 

An alternative approach to determining the PDF is through the Fokker-Planck equation.
The Fokker-Planck equatino is a partial differenial equation (PDE) describing the evolution of the PDF $p(x, t)$ 
associated with $X(t)$. For the SDE given in equation \eqref{eq:sde}, the corresponding Fokker-Planck equation is:

\begin{equation}
    \frac{\partial p(x,t)}{\partial t} = -\frac{\partial}{\partial x}\left[ \mu(x,t) p(x,t) \right] + \frac{1}{2}\frac{\partial^2}{\partial x^2}\left[ \sigma^2(x,t) p(x,t) \right],
\end{equation}

where $p(x,t)$ is the probability density function of the random variable $X_t$. 
Solving this PDE can directly yield the PDF without the sampling noise inherent in Monte Carlo simulations.

For systems involving multiple interacting stochastic processes, we consider a system of coupled SDEs:

\begin{equation}
    d\mathbf{X}_t = \boldsymbol{\mu}(\mathbf{X}_t, t),dt + \boldsymbol{\Sigma}(\mathbf{X}_t, t), d\mathbf{W}_t,
\end{equation}

where $\mathbf{X}_t \in \mathbb{R}^n$ is a vector-valued stochastic process, $\boldsymbol{\mu}(\mathbf{X}_t, t)$ 
is a vector of drift terms, $\boldsymbol{\Sigma}(\mathbf{X}_t, t)$ is a diffusion matrix, and 
$\mathbf{W}_t \in \mathbb{R}^m$ is a vector of independent Wiener processes. The associated Fokker-Planck
equation describing the evolution of the joint probability density $p(\mathbf{x}, t)$ for the system is given by:

\begin{equation}
    \frac{\partial p(\mathbf{x}, t)}{\partial t} = -\nabla_{\mathbf{x}} \cdot \left[ \boldsymbol{\mu}(\mathbf{x}, t) p(\mathbf{x}, t) \right] 
    + \frac{1}{2} \nabla_{\mathbf{x}} \cdot \left( \nabla_{\mathbf{x}} \cdot \left[ \boldsymbol{\Sigma}(\mathbf{x}, t) \boldsymbol{\Sigma}(\mathbf{x}, t)^T p(\mathbf{x}, t) \right] \right).
\end{equation}

where $\nabla_{\mathbf{x}}$ is the gradient operator with respect to the spatial variables $\mathbf{x}$.
Solving this PDE provides a direct method for obtaining the joint probability distributions of systems of 
interacting stochastic systems.