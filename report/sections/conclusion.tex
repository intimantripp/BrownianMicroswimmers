In this report, we investigated several stochastic models aiming to capture the 
boundary-accumulation behavior commonly observed in swimming microorganisms. 
We analyzed three distinct modeling approaches: a purely diffusive model, a model incorporating hydrodynamic interactions with the channel walls, and a piecewise-deterministic Markov process (PDMP) model that explicitly describes boundary capture.

Models 1 and 2, both accounting for cell geometry, provided closely aligned marginal distributions, 
suggesting that hydrodynamic interactions may exert a relatively minor influence under the 
conditions studied here. However, additional exploration of parameter spaces and physical 
scenarios remains essential to thoroughly quantify the role of hydrodynamics in these contexts.

In contrast, Model 3 initially treated cells as point particles, 
yielding exaggerated boundary accumulation. Incorporating geometric considerations into Model 
3 led to a more realistic redistribution of boundary probabilities, yet notable differences 
remained compared to Models 1 and 2. This highlights the importance of accounting explicitly 
for cell geometry in accurately modeling boundary interactions.